...
#include <math.h>
void myDisplay() {
    glClear(GL_COLOR_BUFFER_BIT);
    // ^{\it 색상의 지정}^
    glColor3f(0.0, 1.0, 0.0);    

    // ^{\it 삼각형 그리기로 지정}^
    glBegin(GL_TRIANGLES);
    // mountain 1
    glVertex2f(-0.75, -0.25);
    glVertex2f(0.0, 0.25);
    glVertex2f(0.25, -0.25);
    // mountain 2 - ^{\it 그리기 색상 변경}^
    glColor3f(0.5, 0.5, 0.1);
    glVertex2f(-0.25, -0.25);
    glVertex2f(0.75, 0.25);
    glVertex2f(1.0, -0.25);
    glEnd();

    // ^{\it 사각형 그리기로 지정}^
    glBegin(GL_QUADS);
    // roof ^{\it 파란색으로 색상 변경}^
    glColor3f(0.0, 0.0, 1.0);
    glVertex2f(-1.0, 0.25);
    glVertex2f(-0.75, 0.5);
    glVertex2f(-0.25, 0.5);
    glVertex2f(0.0, 0.25);
    // house ^{\it 노란색으로 색상 변경}^
    glColor3f(1.0, 1.0, 0.0);
    glVertex2f(-0.75, 0.25);
    glVertex2f(-0.75, -0.25);
    glVertex2f(-0.25, -0.25);
    glVertex2f(-0.25, 0.25);
    // tree ^{\it 갈색으로 변경}^
    glColor3f(0.7, 0.5, 0.0);
    glVertex2f(0.5, 0.25);
    glVertex2f(0.75, 0.25);
    glVertex2f(0.75, -0.25);
    glVertex2f(0.5, -0.25);
    glEnd();

    // ^{\it 입력된 정점을 모두 이용하는 다각형 그리기로 지정}^
    glBegin(GL_POLYGON);
    int n=20;
    float radius=0.1;
    glColor3f(1.0, 1.0, 0.0);
    float angle = 0.0; float step=(3.14159*2.0)/n;
    // ^{\it 반복문 내에서 여러 개의 정점 좌표를 계산한 뒤에 지정하는 방식}^
    // ^{\it 여기서는 원을 이루는 정점들을 계산하고 있음}^
    while (angle<3.14159*2.0) {
        glVertex2f(radius*cos(angle), radius*sin(angle)+0.75);
        angle += step;
    }
    glEnd();

    // ^{\it 원의 중심을 옮기고 반지름을 바꾼 뒤에 다시 그림}^
    glBegin(GL_POLYGON);
    n=20;
    radius=0.25;
    glColor3f(0.0, 1.0, 0.0);
    angle = 0.0; step=(3.14159*2.0)/n;
    while (angle<3.14159*2.0) {
        glVertex2f(radius*cos(angle)+0.625, radius*sin(angle)+0.25);
        angle += step;
    } 
    glEnd();
    glFlush();
}